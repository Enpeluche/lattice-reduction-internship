
\section{Notion de degré en ligne}

\begin{comment}
\begin{definition}
    Soit \( M \in \K[x]^{m \x n} \), la \textbf{taille} de \( M \), notée \( \size(M) \), est le nombre de coefficients distincts de \( \K \) nécessaires pour sa représentation dense.

    Et on a la relation
    \[
        \size(M) = \sum_{i,j} \size(a_{i,j}) = \sum_{i,j}( 1 + \max(0, \deg(M_{i,j}))).
    \]
\end{definition}
En général, la taille n'est pas compatible avec le produit matriciel, mais cela peut être le cas dans certains cas particuliers.
\end{comment}

\begin{notation}
    On note \( [d] \) un polynôme de degré \( d \). Par exemple \( x^2 + 1 \) sera noté \( [2] \), \(  x^9 \) sera noté \( [9] \).
\end{notation}

\begin{example}
    La multiplication ne se passe pas forcément bien. On voit que cela donne des bornes non pertinentes et il serait utile de rajoute des critères de mesure du degré. Considérons les matrices de degrés :
    \[
        \begin{pmatrix}
            [100] & [1] \\
            [100] & [1]
        \end{pmatrix}
        \begin{pmatrix}
            [1] & [1] \\
            [1] & [1]
        \end{pmatrix}
        =
        \begin{pmatrix}
            [101] & [101] \\
            [101] & [101]
        \end{pmatrix}
    \]
\end{example}

\begin{definition}
    \leavevmode\vspace{0.5\baselineskip}

    $\bullet \quad$ Pour \( \mathbf{m} = (m_1, \cdots ,m_n) \in \K[x]^{1 \x n} \), on définit son \textbf{degré en ligne} par :
    \[
        \rdeg(\mathbf{m}) = \max_{1 \leq i \leq n} \deg(m_i)\in \Z
    \]
    $\bullet \quad$ Pour \( M =
    \begin{pmatrix}
        \cdots \mathbf{M_1} \cdots \\
        \vdots                     \\
        \cdots \mathbf{M_n} \cdots
    \end{pmatrix},
    \mathbf{M_i} \in \K[x]^{1 \x n}  ~~ \forall 1 \leq i \leq n \), on définit son \textbf{degré en ligne} par :
    \[
        \rdeg(M) = (\rdeg(\mathbf{M_i}))_{1 \leq i \leq n} \in \Z^n
    \]
\end{definition}

\begin{example}
    Soit
    \(
    M =
    \begin{pmatrix}
        3x + 4 & x^9     \\
        5      & x^2 + 1
    \end{pmatrix}
    \in \F_2[x].
    \)
    Alors
    \(
    \rdeg(M)=
    \begin{pmatrix}
        \rdeg(~3x+4~,~x^9~) \\
        \rdeg(~5~,~x^2+1~)
    \end{pmatrix}
    =
    \begin{pmatrix}
        9 \\
        2
    \end{pmatrix}
    \)
\end{example}

Cette définition du degré de ligne présente une limite : si \( c=bA \), on a bien en général \( \rdeg(c)\leq \rdeg(b)+\rdeg(A) \), mais cette majoration est souvent trop lâche pour nos besoins. Ce qui nous intéresse est de pouvoir caractériser plus finement le degré de \( c \), voire d’obtenir une égalité. Cela motive l’introduction de la définition de degré décalé.

\begin{definition}
    Soit \( \vec{\mathbf{s}} =(s_1, \cdots, s_n)\in \Z^n \). On appelle \( \vec{\mathbf{s}} \) le \textbf{vecteur de décalage}.

    $\bullet \quad$ Pour \( \mathbf{m} = (m_1, \cdots ,m_n) \in \K[x]^{1 \x n} \), on définit son \textbf{degré en ligne \( \vec{\mathbf{s}} \)-décalé} par :
    \[
        \rdeg_{\vec{\mathbf{s}}}(\mathbf{m}) = \max_{1 \leq i \leq n} (\deg(m_i) + s_i)
    \]

    $\bullet \quad$ Pour \( M =
    \begin{pmatrix}
        \cdots \mathbf{M_1} \cdots \\
        \vdots                     \\
        \cdots \mathbf{M_n} \cdots
    \end{pmatrix},
    \mathbf{M_i} \in \K[x]^{1 \x n}  ~~ \forall 1 \leq i \leq n \), on définit son \textbf{degré en ligne décalé} par :
    \[
        \rdeg_{\vec{\mathbf{s}}}(M) = \big( \rdeg_{\vec{\mathbf{s}}}(\mathbf{M_i}) \big)_{1 \leq i \leq m} \in \Z^m
    \]
\end{definition}

\begin{notation}
    Soit $\vec{\mathbf{s}}=(s_1, \cdots, s_n) \in \Z^n$. On note $x^{\vec{\mathbf{s}}}$ la matrice diagonale
    \(
    \begin{pmatrix}
        x^{s_1} &        &         \\
                & \ddots &         \\
                &        & x^{s_n}
    \end{pmatrix} \in M_n(\K [x])
    \).
\end{notation}

\begin{proposition}
    Soit \( A \in \K[x]^{m \x n} \), et \( \vec{\mathbf{s}} \in \Z^n \). Alors \( \rdeg_{\vec{\mathbf{s}}}(A) = \rdeg (A x^{\vec{\mathbf{s}}}) \).
\end{proposition}

\begin{example}
    Soit
    \(
    M =
    \begin{pmatrix}
        3x + 4 & x^9     \\
        5      & x^2 + 1
    \end{pmatrix}
    \in \R_2[x]
    \quad \text{et} \quad \vec{\mathbf{s}} =(8, 0)
    \)

    Alors
    \[
        \rdeg_{\vec{\mathbf{s}}} (M) = \rdeg (M\cdot x^{\vec{\mathbf{s}}}) =
        \rdeg
        \begin{pmatrix}
            3x^9 + 4 x^8 & x^9   \\
            5x^8         & x^2+1
        \end{pmatrix}
        =(9, 8)
    \]
\end{example}

\begin{proposition}
    Soit \( A \in \K[x]^{m \x n} \) et \( \vec{\mathbf{s}} \in \Z^n \). Alors, on a les propriétés suivantes :

    $\bullet ~$ \( \rdeg_{\vec{\mathbf{s}}}(A) = \vec{\mathbf{v}} \) si et seulement si \( \rdeg(x^{-\vec{\mathbf{v}}} A x^{\vec{\mathbf{s}}}) = 0 \).

    $\bullet ~$ \( \rdeg_{\vec{\mathbf{s}}}(A) \leq \vec{\mathbf{v}} \) si et seulement si \( \rdeg(x^{-\vec{\mathbf{v}}} A x^{\vec{\mathbf{s}}}) \leq 0 \).

\end{proposition}

\begin{example}
    Soit
    \[
        F =
        \begin{pmatrix}
            1 & 0         & 1     \\
            x & 1         & x + 1 \\
            1 & x^3 + x^2 & x
        \end{pmatrix},
        \quad \vec{\mathbf{u}} = (1, 0, 0, 1).
    \]

    Alors
    \[
        \vec{\mathbf{v}} = \rdeg_{\vec{\mathbf{u}}}(F) = (1,2,3,4)
        \quad \text{et} \quad
        x^{-\vec{\mathbf{v}}} A x^{\vec{\mathbf{s}}} =
        \begin{pmatrix}
            1      & 0          & x^{-1}          \\
            1      & x^{-2}     & x^{-2} + x^{-1} \\
            x^{-2} & x^{-1} + 1 & x^{-2}
        \end{pmatrix}.
    \]
\end{example}

On va définir un ordre sur les degrés de ligne, bien que non total.

\begin{definition}
    Soit \( m \in \N^* \) et soient \( \mathbf{u} = (u_1, \dots, u_m) \), \( \mathbf{v} = (v_1, \dots, v_m) \in \Z^m \) deux vecteurs de degrés de ligne, triés par valeur croissante.
    On définit une relation d'ordre partiel, notée \( \leq_{ob} \) (ordre obtenu par composantes), par :
    \[
        \mathbf{u} \leq_{ob} \mathbf{v} \quad \text{si et seulement si} \quad u_i \leq v_i \quad \text{pour tout } i \in \{1, \dots, m\}.
    \]
\end{definition}

\begin{proposition}
    Soit \( A \in \K[x]^{m \x n} \), \( \vec{\mathbf{b}} \in \K[x]^{1 \x m} \) et \( \vec{\mathbf{c}}=\vec{\mathbf{b}}A \). Soit \( \vec{\mathbf{v}} = \rdeg_{\vec{\mathbf{u}}}(A) \) et \( w = \rdeg_{\vec{\mathbf{v}}}(b) \).

    Alors
    \[
        \rdeg_{\vec{\mathbf{u}}}(c) \leq_{ob} w.
    \]
\end{proposition}

Les notions de degrés introduites précédemment, ainsi que les techniques qui en découlent, permettent d'accélérer certains algorithmes dans des situations spécifiques. Cependant, elles présentent des limites structurelles importantes : notamment, les degrés de lignes et de colonnes ne possèdent pas de bonnes propriétés vis-à-vis de la multiplication matricielle. De plus, plusieurs problèmes restent ouverts quant à la possibilité d'obtenir des algorithmes plus efficaces pour le calcul du déterminant ou de l'inverse de matrices polynomiales. Ces obstacles mettent en évidence l'intérêt crucial de la réduction des matrices polynomiales, outil indispensable pour contrôler la croissance des degrés et améliorer ainsi les performances des algorithmes associés.