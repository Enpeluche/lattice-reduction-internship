\chapter*{Remerciements}

\begin{justify}
    Je tiens en premier lieu à remercier \textbf{Romain}, mon encadrant, pour son accompagnement
    tout au long de ce stage. Toujours patient et bienveillant, même lorsque je disais que
    $a\mathbb{Z} + b\mathbb{Z} = (a+b)\mathbb{Z}$. Il a su créer un environnement stimulant et
    agréable. Ses explications, sa disponibilité et sa bonne humeur ont largement contribué à ma
    progression scientifique et à mon bien-être personnel. Je lui suis sincèrement reconnaissant
    pour son soutien tant humain que rigoureux.
\end{justify}

\vspace{0.2cm}

\begin{justify}
    Un immense merci également à \textbf{Eleonora}, qui avait généreusement accepté d’être mon
    encadrante par intérim, au cas où Romain aurait été indisponible. J’ai pu mesurer toute l'ampleur
    du travail colossal effectué par les chercheurs au quotidien, un véritable engagement et une
    source d'inspiration.
\end{justify}

\vspace{0.2cm}

\begin{justify}
    Je souhaite remercier \textbf{Katharina} pour m'avoir offert l'opportunité de présenter mes
    travaux au Lattice Club. Si l’exercice était stressant, la bienveillance de l'équipe a transformé
    ce moment en une expérience enrichissante et mémorable.
\end{justify}

\vspace{0.2cm}

\begin{justify}
    Un grand merci aussi à M. \textbf{Goncalves} et M. \textbf{Giroudeau} pour leurs retours. Parfois
    piquants mais instructifs, leurs commentaires sont comme la moutarde : on apprend vite à les apprécier
    et à en saisir toute la valeur pédagogique.
\end{justify}

\vspace{0.2cm}

\begin{justify}
    Merci à Mme \textbf{Fourcadier} et à toute l'équipe administrative en charge de la gestion des stages.
    Leur réactivité et leur efficacité rendent le parcours administratif agréable.
\end{justify}

\vspace{0.2cm}

\begin{justify}
    Je tiens à exprimer toute ma gratitude à mes collègues de bureau, \textbf{Quentin}, \textbf{Geoffroy} et
    \textbf{Logan}. Quentin, avec qui j’ai partagé tant de discussions enrichissantes m’a immédiatement inclus
    dans l’équipe. Geoffroy, dont la lucidité remarquable sur le monde académique a toujours donné lieu à des
    échanges stimulants. Logan, quant à lui, véritable incarnation du « chill guy originel », a apporté au
    bureau une sérénité et une bonne humeur constantes.
\end{justify}

\vspace{0.2cm}

\begin{justify}
    Un grand merci également à \textbf{Federico}, véritable rayon de soleil du laboratoire, dont les anecdotes
    et l’amour communicatif de l’Italie ont toujours illuminé mes journées. Merci aussi à \textbf{Laz}, pour
    sa gentillesse permanente.
\end{justify}

\vspace{0.2cm}

\begin{justify}
    Merci à \textbf{Andrei} pour son enthousiasme intellectuel hors-norme et ses réflexions captivantes sur
    la complexité de Kolmogorov, appliquée jusque dans le monde des fourmis. Ces moments d’échange furent
    toujours aussi étonnants qu’amusants. Merci aussi à \textbf{Gabrielle} qui, malgré nos rares échanges,
    m'a offert l'occasion unique de comprendre pleinement un accent anglais !
\end{justify}

\vspace{0.2cm}

\begin{justify}
    Merci à \textbf{Matteo} et \textbf{Paul}, fidèles compagnons des repas du mardi midi.
\end{justify}

\vspace{0.2cm}

\begin{justify}
    Un clin d’œil particulier aux personnes du foodtruck, surtout par ces fortes chaleurs.
\end{justify}

\vspace{0.2cm}

\begin{justify}
    Je ne saurais conclure ces remerciements sans mentionner les chats du laboratoire, maîtres incontestés des
    lieux, ainsi que l’incontournable machine à café. Fidèle alliée de mes neurones.
\end{justify}

\vspace{0.2cm}

\begin{justify}
    Enfin, ma reconnaissance la plus profonde va à \textbf{Julie}, ma compagne. Son écoute et sa patience
    ont été essentiels, sa présence a été mon plus précieux soutien.
\end{justify}

\vspace{0.2cm}

\begin{justify}
    Toutes ces interactions, grandes ou petites, ont contribué significativement à mon équilibre personnel
    pendant cette période intense. Souvent, ce sont ces gestes simples qui ont l'impact le plus fort sur le
    bien-être quotidien. Merci infiniment à toutes et tous.
\end{justify}