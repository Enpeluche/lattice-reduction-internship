
\section{Terminaison et complexité}
\begin{theoreme}[Terminaison et complexité]
    On pose \( \displaystyle A = \max_{1 \leq i \leq n} \| \bg_i \| \). L'algorithme \hyperref[algo:LLL_MCA]{\emph{LLL}} termine et utilise \( \OO(n^4 \log A) \) opérations arithmétiques sur des entiers.
\end{theoreme}

La difficulté est de montrer que la boucle Tant que ne va pas s'exécuter indéfiniment.

\begin{lemma}
    \leavevmode\vspace{0.3\baselineskip}
    \begin{enumerate}
        \item Orthogonalisation de Gram-Schmidt nécessite \( \OO(n^3) \) opérations dans \( \Z \).

        \item \hyperref[step:P]{\emph{Proprification de \( \bg_i \)}} nécessite \( \OO(n^2) \) opérations dans \( \Z \).

        \item \hyperref[step:R]{\emph{Réduction de \( \bg_{i-1}, \bg_{i} \)}} nécessite \( \OO(n) \) opérations dans \( \Z \).
    \end{enumerate}
\end{lemma}

Il reste à borner le nombre d'itérations de la boucle Tant que à l'étape \hyperref[step:LLL]{\emph{LLL}}.

Pour tout \( 1 \leq k \leq n \), on pose
\[
    \bg_k =
    \begin{pmatrix}
        \bg_1  \\
        \vdots \\
        \bg_k
    \end{pmatrix}
    \in \Z^{k \x n}
    , \quad d_0=1, \quad d_k = \det(\bg_k \cdot \bg_k^T) \in \Z.
\]

\begin{lemma}
    Pour tout \( 1 \leq k \leq n \), on a :
    \[
        d_k = \prod_{1 \leq l \leq k} \| \bg_l^* \|^2 > 0.
    \]
\end{lemma}

\begin{lemma}
    \leavevmode\vspace{0.5\baselineskip}
    \begin{enumerate}
        \item \hyperref[step:P]{\emph{Proprification de \( \bg_i \)}} ne change pas \( d_k \)  pour tout \( 1 \leq k \leq n \).

        \item Si \( \bg_{i-1} \) et \( \bg_i \) sont échangés à l’étape \hyperref[step:P]{\emph{Réduction de \( \bg_{i-1}, \bg_{i} \)}}, et si \( d_k^* \) désigne la nouvelle valeur de \( d_k \), alors :
              \[
                  d_k^* = d_k \quad \text{pour tout } k \neq i - 1, \quad \text{et} \quad d_{i-1}^* \leq \frac{3}{4} d_{i-1}.
              \]
    \end{enumerate}
\end{lemma}

\begin{proof}
    \begin{enumerate}
        \item D'après le lemme 2.2 \hyperref[step:P]{\emph{Proprification de \( \bg_i \)}} ne modifie pas \( \bg_k^* \) et donc ne modifie pas \( d_k \).

        \item Pour \( k \neq i-1\), une exécution de \hyperref[step:P]{\emph{Réduction de \( \bg_{i-1}, \bg_{i} \)}} multiplie \( \bg_k \) par une matrice de permutation, donc \( d_k^* = d_k \)

              De plus, on a
              \[
                  d_{i-1} \eqjust{2.6} \prod_{1 \leq l \leq i-1} \| \bg_l^* \|^2 \leqjust{2.3} \frac{3}{4} \prod_{1 \leq l \leq i-1} \| \mathbf{h}_l^* \|^2 \eqjust{2.6} \frac{3}{4} d_{i-1}^*
              \]
    \end{enumerate}
\end{proof}
On pose
\[
    D = \prod_{1 \leq k < n} d_k, \quad \displaystyle A = \max_{1 \leq i \leq n} \| \bg_i \|
\]

On désigne \( D_0 \) désigne la valeur de \( D \) au début de l'algorithme, on a \( 1 \leq D \in \Z \) et

\[
    \begin{aligned}
        D_0 & = \|\bg^*_1\|^{2(n-1)} \|\bg^*_2\|^{2(n-2)} \cdots \|\bg^*_{n-1}\|^2 \\
            & \leq \|\bg_1\|^{2(n-1)} \|\bg_2\|^{2(n-2)} \cdots \|\bg_{n-1}\|^2    \\
            & \leq A^{n(n-1)}
    \end{aligned}
\]


Puisque \( g^*_i \) est une projection de \( g_i \) pour tout \( i \).

\begin{lemma}
    \leavevmode\vspace{0.5\baselineskip}
    \begin{enumerate}
        \item \hyperref[step:P]{\emph{Proprification de \( \bg_i \)}} ne modifie pas \( D \).
        \item \( D \) diminue d’au moins un facteur \( 3/4 \) si un échange a lieu dans \hyperref[step:R]{\emph{Réduction de \( \bg_{i-1}, \bg_{i} \)}}
    \end{enumerate}
\end{lemma}

\begin{proof}
    \begin{enumerate}
        \item D'après le lemme 2.7 \hyperref[step:P]{\emph{Proprification de \( \bg_i \)}} ne modifie pas \( d_k \) et donc ne modifie pas \( D \).

        \item Si \( \bg_{i-1} \) et \( \bg_i \) sont échangés lors de l'exécution de \hyperref[step:R]{\emph{Réduction de \( \bg_{i-1}, \bg_{i} \)}}, en notant \( D^* \) la nouvelle valeur de \( D \), alors d'après le lemme 2.7

              \[
                  d_k^* = d_k, \quad d_{i-1}^* \leq \frac{3}{4} d_{i-1} \text{ donc } D^* \leq \frac{3}{4} D.
              \]
    \end{enumerate}
\end{proof}

À tout moment de l’algorithme, soit \( e \in \N \) le nombre d’échanges effectués jusqu’à présent, et \( e^* \) le nombre de fois où la branche alternative (le \textit{else}) dans \hyperref[step:R]{\emph{Réduction de \( \bg_{i-1}, \bg_{i} \)}} a été prise.

\begin{lemma}
    On a
    \[
        e \leq \log_{4/3} D_0 \in \OO(n^2 \log A)
    \]
\end{lemma}

\begin{proof}
    Soit \( D_e \) la valeur de \( D \) après \( e \) échanges.

    On doit avoir
    \[
        1 \;\le\; D_e \; \le\;\left(\frac34\right)^{e} D_0 \le\; \left(\frac34\right)^{e}A^{\,n(n-1)}.
    \]

    En appliquant \( lo\bg_{3/4} \), aux extrémités de l'inégalité.

    \[
        0=\log_{3/4}(1) \;\ge\; e + \log_{3/4}(A^{\,n(n-1)})=e + n(n-1)\frac{\log A}{\log(3/4)}.
    \]

    On en déduit que \( e \; \leq n(n-1)\frac{\log A}{-\log(3/4)}\) et donc \( e \in \OO(n^2 \log A)\)

\end{proof}

\begin{proof}[Preuve de la terminaison et la complexité]

    Comme \( i \) est décrémenté de \( 1 \) lors d’un échange et incrémenté de \( 1 \) sinon l'entier \( i + e - e^* \) est constant tout au long de \hyperref[step:LLL]{\emph{LLL}}.

    Initialement \( i + e - e^* = 2 \) et à la fin de \hyperref[step:LLL]{\emph{LLL}} on a \( n + 1 + e - e^* = 2 \).
    On en déduit donc que \( e + e^* = 2e + n - 1 \in \OO(n^2 \log A) \).
    et donc d'après le lemme 2.5 le coût total de \hyperref[step:LLL]{\emph{LLL}} est \( \OO(n^2 \x n^2 \log A) \) opérations dans \( \Z \). Ce qui acheve la preuve.
\end{proof}
