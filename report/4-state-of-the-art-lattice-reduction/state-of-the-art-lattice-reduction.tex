\chapter*{État de l'art de la réduction de réseaux euclidiens}
\addcontentsline{toc}{chapter}{État de l'art de la réduction de réseaux euclidiens et l'objectif de mon stage}

Voici une rétrospective structurée des avancées majeures dans la réduction de réseaux euclidiens.

\begin{itemize}
    \item[\textbf{1982}] \textbf{LLL} (Lenstra, Lenstra, Lovász) \parencite{Lenstra1982} introduisent le premier algorithme de réduction polynomial, basé sur une combinaison de \emph{size reduction} et d'une condition de Lovász. Il garantit :
          \[
              \|\bb_1\| \leq (4/3)^{(n-1)/2} \cdot \lambda_1(\LL), \quad \text{avec complexité binaire } \mathcal{O}(n^5 \beta^2)
          \]

    \item[\textbf{1991}] \textbf{BKZ} \parencite{Schnorr1994} introduit une approche par blocs. L’algorithme applique un solveur SVP de petite dimension \(\beta\) à des sous-blocs de la base :
          \[
              \|\bb_1\| \leq \gamma_\beta^{(n-1)/(\beta - 1)} \cdot \lambda_1(\LL)
          \]
          Le coût est exponentiel en \(\beta\) mais reste efficace pour \(\beta \leq 40\) en pratique.

    \item[\textbf{2009}] \textbf{L2} \parencite{Nguyen2009} améliore LLL sur le plan de la stabilité numérique, sans gain théorique majeur sur la qualité de la base. Complexité similaire à LLL, mais plus efficace pour des entrées en flottants.

    \item[\textbf{2011}] \textbf{$\tilde{L}_1$} \parencite{Novocin2011} propose une version rapide de LLL inspirée du GCD rapide de Knuth–Schönhage. Il introduit une stratégie récursive appelée \emph{Lift-Reduction} et atteint une complexité quasi-linéaire :
          \[
              \mathcal{O}\left(d^{5+\varepsilon} \beta + d^{\omega+1+\varepsilon} \beta^{1+\varepsilon} \right)
          \]
          avec une qualité comparable à LLL : \( \|\bb_1\| \leq 2^{\alpha n} \cdot |\LL|^{1/n} \).

    \item[\textbf{2011}] \textbf{Terminating BKZ}  \parencite{cryptoeprint:2011/198} propose une modélisation dynamique affine de BKZ. Ils montrent que même interrompu prématurément, BKZ garantit :
          \[
              \|\bb_1\| \leq 2^{\frac{\gamma_\beta(n-1)}{2(\beta - 1)} + \frac{3}{2}} \cdot |\LL|^{1/n}
          \]
          après seulement \( \mathcal{O}(n^3/\beta^2 \cdot \log \|B\|) \) appels à un solveur SVP.

    \item[\textbf{2019}] \textbf{KEF} \parencite{Kirchner2021}propose un algorithme heuristique récursif exploitant la \emph{Geometric Series Assumption} (GSA) pour guider la réduction. Il utilise des techniques de FFT et obtient une complexité heuristique :
          \[
              \widetilde{\mathcal{O}}(n^\omega \cdot \log \kappa(B))
          \]
          avec une qualité empirique équivalente à BKZ en grande dimension (\( n > 2000 \)).

    \item[\textbf{2023}] \textbf{Iterated Compression} \parencite{Ryan2023} présente un algorithme récursif fondé sur des opérations de compression stables, une métrique de \emph{drop}, et des profils dynamiques :
          \[
              \|\bb_1\| \leq 2^{\alpha n} \cdot |\LL|^{1/n}, \quad \|\bb_n^*\| \geq 2^{-\alpha n} \cdot |\LL|^{1/n}
          \]
          avec complexité heuristique :
          \[
              \mathcal{O}(n^\omega(C + n)^{1+\varepsilon}), \quad C = \log(\|B\| \cdot \|B^{-1}\|)
          \]

\end{itemize}